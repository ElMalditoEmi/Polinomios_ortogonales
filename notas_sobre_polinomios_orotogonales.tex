\documentclass{article}

%--- MATH SYMBOLS -------
\usepackage{amsmath}
\usepackage{amssymb}
%-----------------------
\usepackage{imakeidx}
\usepackage{blindtext}

%--- Cajitas facheras ---
\usepackage{tikz}
\usetikzlibrary{backgrounds}
\usepackage[skins]{tcolorbox}
\tikzstyle{background rectangle}=[thin,draw=black]
\newtcolorbox{myframe}[2][]{%
  enhanced,colback=white,colframe=black,coltitle=black,
  sharp corners,boxrule=0.4pt,
  fonttitle=\itshape,
  attach boxed title to top left={yshift=-0.3\baselineskip-0.4pt,xshift=2mm},
  boxed title style={tile,size=minimal,left=0.5mm,right=0.5mm,
    colback=white,before upper=\strut},
  title=#2,#1
}
%------------------------

\makeindex
\title{Notas sobre polinomios ortogonales}
\author{Emilio Pereyra}
\begin{document}
	\begin{titlepage}
	\maketitle
	Algunas notas sobre mi aprendizaje de los polinomios ortogonales que necesitaba 
	para leer cierto paper.
	\end{titlepage}

	\section{Integral Stieltjes-Lebesgue}
	\subsection{Antes: Integral de Riemann-Stieltjes}
	\begin{flushleft}
		La integral de Riemann-Stieltjes es una generalización del concepto
		integral que propuso Riemann.
		Esta integral de R-S toma 2 funciones, llamadas el \textbf{integrando}
		y el \textbf{integrador}.\\
		Veamos un ejemplo donde $f$ es el integrando y la funcion $\alpha$ 
		es el integrador.
		\begin{equation}
			\begin{aligned}
				\int_{a}^{b}f d \alpha
			\end{aligned}
		\end{equation}
		Sea $P=\{x_0,x_1,x_2...,x_n\}$ una partición de $[a,b]$ con 
		$a=x_0<x_1<x_2<...<x_n=b$ .Llamaremos suma Riemman-Stieltjes a:
		\begin{equation}
			\begin{aligned}
				\sum_{k=1}^{n}f(t_k)( \alpha(x_{k}) -  \alpha(x_{k-1})).
			\end{aligned}
		\end{equation}
		$f$ es Riemman-Stieltjes integrable respecto a $\alpha$ ,  en $[a,b]$ si existe 
		un número $A$, tal que para todo $ \varepsilon \in \mathbb{R}_{>0}$, existe $ 
		 P_{\varepsilon}$ una partición con que para toda particion $P$ más fina que 
		 $P_{\varepsilon}$ y cualquier eleccion de los $t_{k}$ tenemos:
		 \begin{equation}
		 	\begin{aligned}
		 		|S(P,f,\alpha)-A| < \varepsilon
		 	\end{aligned}
		 \end{equation}
		 Además notemos que si $ \alpha $  es la función identidad, esto es en realidad una
		 integral de Riemann convencional:
		 \begin{equation}
		 	\begin{aligned}
				\int_{a}^{b}fdx &= \int_{a}^{b}fd \alpha(x)
			\end{aligned}
		 \end{equation}
	\end{flushleft}
	\begin{myframe}{Propiedades}
		1) Es lineal con respecto al integrando
		\begin{equation}
			\begin{aligned}
				 \int_{a}^{b}(c_1f(x)+c_2g(x))d \alpha(x) &= c_1\int_{a}^{b}f(x)d \alpha(x) + c_1\int_{a}^{b}g(x)d \alpha(x)
			\end{aligned}
		\end{equation}
		Y con respecto al integrador:
		\begin{equation}
			\begin{aligned}
				\int_{a}^{b}f(x)d(c_1 \alpha(x) + c_2\beta(x)) = c_1 \int_{a}^{b}f(x)d \alpha(x) + \int_{a}^{b}f(x) d \beta(x)
			\end{aligned}
		\end{equation}

		2) Se puede partir el intervalo de integración. Sea $c \in [a,b]$.
		\begin{equation}
			\begin{aligned}
			 \int_{a}^{b}f(x)d \alpha(x)= \int_{a}^{c}f(x)d \alpha(x) + \int_{c}^{b}f(x)d \alpha(x).
			\end{aligned}
		\end{equation}
		3) Hay integracion por partes.
		\begin{equation}
			\begin{aligned}
				\int_{a}^{b}fd \alpha(x) + \int_{a}^{b} \alpha(x)df(x) = f(b) \alpha(x)- f(a) \alpha(a).
			\end{aligned}
		\end{equation}
	\end{myframe}
	
	\subsubsection{Ejemplos y aplicaciones}
	\underline {Ejemplo 1}:Si $ \alpha(x)$ es una función escalonada, en este caso usaremos:
	\[
	\alpha(x) =
	\begin{cases}
		a \text{ si } x < x_0\\
		c \text{ si } x = x_0\\
		b \text{ si } x > x_0
	\end{cases}
	\] 	
	Entonces los sumandos de de $S(P,f,\alpha)$ son 0, excepto en el sub intervalo $[u,v]$ de 
	$P$, que tiene  $x_0$ como \textbf{punto interior}. Al obtener el limite de la suma,
	tenemos:
	\begin{equation}
		\begin{aligned}
			(b-a)f(x_0)
		\end{aligned}
	\end{equation}
	\underline {Ejemplo 2}:
	Otro ejempmlo es la función parte entera $y=[x]$, que nos ayuda a definir la sumatoria
	a partir de una integral:
	\begin{equation}
		\begin{aligned}
			\sum_{k=1}^{n}f(k) = \int_{0}^{n}d[x], n \in \mathbb{N}
		\end{aligned}
	\end{equation}
	\center{Notar: \fbox{$[t_k] - [t_{k-1}] = 0$ si son un tramo con la misma parte entera}}
	\\
	\underline {Ejemplo 2}:
	Como ultimo ejemplo vamos a hablar de como funciona la integral R-S sobre curvas:\\
	Una curva $f(x,y,z)$, respecto a la longitud de arco de una curva:
	\begin{equation}
		\begin{aligned}
			\vec{r}^{\ } = x(t)\vec{i}^{\ } + y(t) \vec{j}^{\ } +
			z(t) \vec{k}^{\ }\text{ } \text{con } t_0 < t < t_f.
		\end{aligned}
	\end{equation}
	Esto puede ser visto como una integral de Riemann-Stieltjes en que el integrando es:
	\begin{equation}
		\begin{aligned}
			F(t) = f(x(t),y(t),z(t))
		\end{aligned}
	\end{equation}
	Y el integrador es la funcion  $S(t)$ que indica la longitud de arco de curva.
	 \begin{equation}
		\begin{aligned}
			\int_{\Gamma}^{}f(f,y,z) ds &= \int_{t_0}^{t_f}f(x(t),y(t),z(t))ds(t)\\
						    &= \int_{t_0}^{t_f}F(t)s'(t)dt
		\end{aligned}
	\end{equation}

	\subsection{Teoria de medida (para comprender la integral de Lebesgue)}
	Una medida de un conjunto es la forma rigurosa de asignar un numero a cada
	subconjunto apropiado de un conjunto.
\end{document}
