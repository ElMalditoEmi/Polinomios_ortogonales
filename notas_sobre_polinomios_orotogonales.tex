\documentclass{article}
\usepackage{amsmath}
\usepackage{amssymb}
\usepackage{imakeidx}
\usepackage{blindtext}
\makeindex
\title{Notas sobre polinomios ortogonales}
\author{Emilio Pereyra}
\begin{document}
	\begin{titlepage}
	\maketitle
	Algunas notas sobre mi aprendizaje de los polinomios ortogonales que necesitaba 
	para leer cierto paper.
	\end{titlepage}

	\section{Integral Stieltjes-Lebesgue}
	\subsection{Antes: Integral de Riemann-Stieltjes}
	\begin{flushleft}
		La integral de Riemann-Stieltjes es una generalización del concepto
		integral que propuso Riemann.
		Esta integral de R-S toma 2 funciones, llamadas el \textbf{integrando}
		y el \textbf{integrador}.\\
		Veamos un ejemplo donde $f$ es el integrando y la funcion $\alpha$ 
		es el integrador.
		\begin{equation}
			\begin{aligned}
				\int_{a}^{b}f d \alpha
			\end{aligned}
		\end{equation}
		Sea $P=\{x_0,x_1,x_2...,x_n\}$ una partición de $[a,b]$ con 
		$a=x_0<x_1<x_2<...<x_n=b$ .Llamaremos suma Riemman-Stieltjes a:
		\begin{equation}
			\begin{aligned}
				\sum_{k=1}^{n}f(t_k)( \alpha(x_{k}) -  \alpha(x_{k-1})).
			\end{aligned}
		\end{equation}
		$f$ es Riemman-Stieltjes integrable respecto a $\alpha$ ,  en $[a,b]$ si existe 
		un número $A$, tal que para todo $ \varepsilon \in \mathbb{R}_{>0}$, existe $ 
		 P_{\varepsilon}$ una partición con que para toda particion $P$ más fina que 
		 $P_{\varepsilon}$ y cualquier eleccion de los $t_{k}$ tenemos:
		 \begin{equation}
		 	\begin{aligned}
		 		|S(P,f,\alpha)-A| < \varepsilon
		 	\end{aligned}
		 \end{equation}
		 Además notemos que si $ \alpha $  es la función identidad, esto es en realidad una
		 integral de Riemann convencional:
		 \begin{equation}
		 	\begin{aligned}
				\int_{a}^{b}fdx &= \int_{a}^{b}fd \alpha(x)
			\end{aligned}
		 \end{equation}
	\end{flushleft}	
\end{document}
